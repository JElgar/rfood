% The document class supplies options to control rendering of some standard
% features in the result.  The goal is for uniform style, so some attention 
% to detail is *vital* with all fields.  Each field (i.e., text inside the
% curly braces below, so the MEng text inside {MEng} for instance) should 
% take into account the following:
%
% - author name       should be formatted as "FirstName LastName"
%   (not "Initial LastName" for example),
% - supervisor name   should be formatted as "Title FirstName LastName"
%   (where Title is "Dr." or "Prof." for example),
% - degree programme  should be "BSc", "MEng", "MSci", "MSc" or "PhD",
% - dissertation title should be correctly capitalised (plus you can have
%   an optional sub-title if appropriate, or leave this field blank),
% - dissertation type should be formatted as one of the following:
%   * for the MEng degree programme either "enterprise" or "research" to
%     reflect the stream,
%   * for the MSc  degree programme "$X/Y/Z$" for a project deemed to be
%     X%, Y% and Z% of type I, II and III.
% - year              should be formatted as a 4-digit year of submission
%   (so 2014 rather than the academic year, say 2013/14 say).
                 

\documentclass[ oneside,% the name of the author
                    author={James Elgar},
                % the degree programme: BSc, MEng, MSci or MSc.
                    degree={MEng},
                % the dissertation    title (which cannot be blank)
                     title={Bidirectional transformer between functional and \\ object-oriented programming in Rust},
                % the dissertation subtitle (which can    be blank)
                  subtitle={}]{dissertation}
                  
% Commands
\usepackage{hyperref}
\newcommand{\weixin}{Anonymous }

\usepackage{listings}
\usepackage{xcolor}

\usepackage{minted}
\newcommand{\rust}[1]{\mintinline{rust}{#1}}
\newcommand{\rustfile}[1]{\inputminted[xleftmargin=20pt,linenos]{rust}{../src/#1}}

\begin{document}

% =============================================================================

% This macro creates the standard UoB title page by using information drawn
% from the document class (meaning it is vital you select the correct degree 
% title and so on).

\maketitle

% After the title page (which is a special case in that it is not numbered)
% comes the front matter or preliminaries; this macro signals the start of
% such content, meaning the pages are numbered with Roman numerals.

\frontmatter


%\lstlistoflistings

% The following sections are part of the front matter, but are not generated
% automatically by LaTeX; the use of \chapter* means they are not numbered.

% -----------------------------------------------------------------------------

\chapter*{Abstract}

% TODO I think is should really be in the intro, focus more on what actually happens
There are benefits to using both object orient and functional programming. With modern programming languages adopting traditional functional feature, developer often have the choice between the two paradigms. However once a paradigm is selected switching can be a time consuming process. In \cite{food} \weixin proposes a framework for transforming between these decomposition styles. This paper explorers the generality of this approach through an implementation of the transformer in rust. 

The key contributions of this paper are:

\begin{quote}
\noindent
\begin{itemize}
    \item Created an implementation of the \cite{food} transformer in rust
    \item Extend the \cite{food} transformer to support a basic implementation of the rust type system
    \item Explored extensions to the transformer including generics and mutability
\end{itemize}
\end{quote}

% {\bf A compulsory section, of at most 300 words} 
% \vspace{1cm} 

% ---- Reference/Example ---- %
% \noindent
% This section should pr\'{e}cis the project context, aims and objectives,
% and main contributions (e.g., deliverables) and achievements; the same 
% section may be called an abstract elsewhere.  The goal is to ensure the 
% reader is clear about what the topic is, what you have done within this 
% topic, {\em and} what your view of the outcome is.

% The former aspects should be guided by your specification: essentially 
% this section is a (very) short version of what is typically the first 
% chapter. If your project is experimental in nature, this should include 
% a clear research hypothesis.  This will obviously differ significantly
% for each project, but an example might be as follows:

% \begin{quote}
% My research hypothesis is that a suitable genetic algorithm will yield
% more accurate results (when applied to the standard ACME data set) than 
% the algorithm proposed by Jones and Smith, while also executing in less
% time.
% \end{quote}

% \noindent
% The latter aspects should (ideally) be presented as a concise, factual 
% bullet point list.  Again the points will differ for each project, but 
% an might be as follows:

% \begin{quote}
% \noindent
% \begin{itemize}
% \item I spent $120$ hours collecting material on and learning about the 
%       Java garbage-collection sub-system. 
% \item I wrote a total of $5000$ lines of source code, comprising a Linux 
%       device driver for a robot (in C) and a GUI (in Java) that is 
%       used to control it.
% \item I designed a new algorithm for computing the non-linear mapping 
%       from A-space to B-space using a genetic algorithm, see page $17$.
% \item I implemented a version of the algorithm proposed by Jones and 
%       Smith in [6], see page $12$, corrected a mistake in it, and 
%       compared the results with several alternatives.
% \end{itemize}
% \end{quote}

% -----------------------------------------------------------------------------


\chapter*{Dedication and Acknowledgements}

{\bf A compulsory section}
\vspace{1cm} 

\noindent
It is common practice (although totally optional) to acknowledge any
third-party advice, contribution or influence you have found useful
during your work.  Examples include support from friends or family, 
the input of your Supervisor and/or Advisor, external organisations 
or persons who  have supplied resources of some kind (e.g., funding, 
advice or time), and so on.

% -----------------------------------------------------------------------------

% This macro creates the standard UoB declaration; on the printed hard-copy,
% this must be physically signed by the author in the space indicated.

\makedecl



% -----------------------------------------------------------------------------

% LaTeX automatically generates a table of contents, plus associated lists 
% of figures and tables.  These are all compulsory parts of the dissertation.

\tableofcontents
\listoffigures
\listoftables

% -----------------------------------------------------------------------------



\chapter*{Ethics Statement}

This project did not require ethical review, as determined by my supervisor, [fill in name].

% -----------------------------------------------------------------------------

% \noindent
% This section should present a detailed summary, in bullet point form, 
% of any third-party resources (e.g., hardware and software components) 
% used during the project.  Use of such resources is always perfectly 
% acceptable: the goal of this section is simply to be clear about how
% and where they are used, so that a clear assessment of your work can
% result.  The content can focus on the project topic itself (rather,
% for example, than including ``I used \mbox{\LaTeX} to prepare my 
% dissertation''); an example is as follows:

\chapter*{Supporting Technologies}

\noindent
To implement the transformer I used the following technologies:

\begin{quote}
\noindent
\begin{itemize}
\item I used the Rust programming language to implement the transformer. \url{https://www.rust-lang.org/}
\item I used the syn package to create and parse the AST. \url{https://docs.rs/syn/latest/syn/}
\item I used the quote package to covert the transformed AST back to rust code. \url{https://docs.rs/quote/latest/quote/}
\item I used rustfmt (a component of the Rust language) to format the transformed output code.
\end{itemize}
\end{quote}

% -----------------------------------------------------------------------------

\chapter*{Notation and Acronyms}

{\bf An optional section}
\vspace{1cm} 

\noindent
Any well written document will introduce notation and acronyms before
their use, {\em even if} they are standard in some way: this ensures 
any reader can understand the resulting self-contained content.  

Said introduction can exist within the dissertation itself, wherever 
that is appropriate.  For an acronym, this is typically achieved at 
the first point of use via ``Advanced Encryption Standard (AES)'' or 
similar, noting the capitalisation of relevant letters.  However, it 
can be useful to include an additional, dedicated list at the start 
of the dissertation; the advantage of doing so is that you cannot 
mistakenly use an acronym before defining it.  A limited example is 
as follows:

\begin{quote}
\noindent
\begin{tabular}{lcl}
AST                 &:     & Abstract Syntax Tree                                        \\
OOP                 &:     & Object Oriented Programming                                 \\
FP                  &:     & Functional Programming                                      \\
FOOD                &:     & Functional and Object Oriented Decomposition                \\
\end{tabular}
\end{quote}


% =============================================================================

% After the front matter comes a number of chapters; under each chapter,
% sections, subsections and even subsubsections are permissible.  The
% pages in this part are numbered with Arabic numerals.  Note that:
%
% - A reference point can be marked using \label{XXX}, and then later
%   referred to via \ref{XXX}; for example Chapter\ref{chap:context}.
% - The chapters are presented here in one file; this can become hard
%   to manage.  An alternative is to save the content in seprate files
%   the use \input{XXX} to import it, which acts like the #include
%   directive in C.

\mainmatter


\chapter{Introduction}
\label{chap:context}

% {\bf Unlike the frontmatter up to and including the Summary of Changes, which you should not deviate from, Chapters 1--5 represent a suggested outline only. This outline will only be appropriate for a specific type of project. You should talk with your supervisor about the best way to structure your own dissertation, but ultimately the choice is yours. However, almost every project will want to include the content discussed in these chapters in some way. For more advice on structuring your dissertation, see the unit handbook.}
% \vspace{1cm} 

% \noindent
% This chapter should introduce the project context and motivate each of the proposed aims and objectives.  Ideally, it is written at a fairly high-level, and easily understood by a reader who is technically competent but not an expert in the topic itself.

% In short, the goal is to answer three questions for the reader.  First, what is the project topic, or problem being investigated?  Second, why is the topic important, or rather why should the reader care about it?  For example, why there is a need for this project (e.g., lack of similar software or deficiency in existing software), who will benefit from the project and in what way (e.g., end-users, or software developers) what work does the project build on and why is the selected approach either important and/or interesting (e.g., fills a gap in literature, applies results from another field to a new problem).  Finally, what are the central challenges involved and why are they significant? 
 
% The chapter should conclude with a concise bullet point list that summarises the aims and objectives.  For example:

% \begin{quote}
% \noindent
% The high-level objective of this project is to reduce the performance 
% gap between hardware and software implementations of modular arithmetic.  
% More specifically, the concrete aims are:

% \begin{enumerate}
% \item Research and survey literature on public-key cryptography and
%       identify the state of the art in exponentiation algorithms.
% \item Improve the state of the art algorithm so that it can be used
%       in an effective and flexible way on constrained devices.
% \item Implement a framework for describing exponentiation algorithms
%       and populate it with suitable examples from the literature on 
%       an ARM7 platform.
% \item Use the framework to perform a study of algorithm performance
%       in terms of time and space, and show the proposed improvements
%       are worthwhile.
% \end{enumerate}
% \end{quote}

\noindent
With the rapid adoption of functional and object oriented features in modern programming languages, developers now have the choice of which paradigm to use in different situations. Both paradigms come with distinct advantages and disadvantages, and such the correct choice for a given feature can often be a tricky decision. In any case a "correct" decision may not be possible as both paradigms could have different values in the same feature. If a developer makes a choice they later regret, refactoring to another paradigm can be an extremely slow process.

Therefore, to truly harness the value of both paradigms, in \cite{food} \weixin explores the idea of a transformer, which allows developer to automatically transform between these two paradigms. This allows developers to instantly switch the style of there code depending on the task.

\weixin proposes a framework to transform between these two decomposition styles and create an implementation in Scala to demo its functionality.

The goal of this paper is to replicate this implementation in Rust. Rust was created by the Mozilla foundation with the intention of replacing C++ in their stack. It has a strong focus on performance and safety which it achieves partly through its borrow checker. With Rust being a low level language it exposes concepts to the developer which are not available in Scala, such as references and borrows. This requires extension to the existing transformer to allow it to transform with these concepts. 

% -----------------------------------------------------------------------------

% \noindent
% This chapter is intended to describe the background on which execution of the project depends. This may be a technical or a contextual background, or both. The goal is to provide a detailed explanation of the specific problem at hand, and existing work that is relevant (e.g., an existing algorithm that you use, alternative solutions proposed, supporting technologies).  

% Per the same advice in the handbook, note there is a subtly difference from this and a full-blown literature review (or survey).  The latter might try to capture and organise (e.g., categorise somehow) \emph{all} related work, potentially offering meta-analysis, whereas here the goal is simple to ensure the dissertation is self-contained.  Put another way, after reading this chapter a non-expert reader should have obtained enough background to understand what \emph{you} have done (by reading subsequent sections), then accurately assess your work against existing relevant related work.  You might view an additional goal as giving the reader confidence that you are able to absorb, understand and clearly communicate highly technical material and to situate your work within existing literature.

\chapter{Background}
\label{chap:technical}

% Following on from this dudes work with rust
The work of \weixin in \cite{food} focuses on the transformation of the

\section{Decomposition}

Decomposition in computer science, is how a problem is broken down into smaller parts. Object oriented programming uses classes or objects for this whilst functional programming uses functions and data types.

The pure OOP style described by Cook in \cite{cook} and used by \weixin to structure the input of the code, maps closely to the object style in Rust.

In pure OOP interfaces are used as types. Interfaces describe required operations for the type. Classes which define these operations are then implementations of this interface.

In Rust, interfaces are implemented as traits, whilst classes are implemented as structs. Structs can then explicitly implement traits with an impl as shown in \ref{fig:impl-example}

\rustfile{main.rs}
% \lstinputlisting[language=Rust]{../src/main.rs}

\section{Rust Type System}



% -----------------------------------------------------------------------------

% \noindent
% This chapter is intended to describe what you did: the goal is to explain
% the main activity or activities, of any type, which constituted your work 
% during the project.  The content is highly topic-specific, but for many 
% projects it will make sense to split the chapter into two sections: one 
% will discuss the design of something (e.g., some hardware or software, or 
% an algorithm, or experiment), including any rationale or decisions made, 
% and the other will discuss how this design was realised via some form of 
% implementation.

% This is, of course, far from ideal for {\em many} project topics.  Some
% situations which clearly require a different approach include:

% \begin{itemize}
% \item In a project where asymptotic analysis of some algorithm is the goal,
%       there is no real ``design and implementation'' in a traditional sense
%       even though the activity of analysis is clearly within the remit of
%       this chapter.
% \item In a project where analysis of some results is as major, or a more
%       major goal than the implementation that produced them, it might be
%       sensible to merge this chapter with the next one: the main activity 
%       is such that discussion of the results cannot be viewed separately.
% \end{itemize}

% \noindent
% Note that it is common to include evidence of ``best practice'' project 
% management (e.g., use of version control, choice of programming language 
% and so on).  Rather than simply a rote list, make sure any such content 
% is useful and/or informative in some way: for example, if there was a 
% decision to be made then explain the trade-offs and implications 
% involved.


\chapter{Project Execution}
\label{chap:execution}

\section{Setup}

Before any transformations take place, the provided code is transformed into an abstract syntax tree (AST). This provides a concrete format to represent the code syntax, keeping the transformer separate from the code parsing stage. Rust provides access to the \verb|rustc| package, which is its compiler. This contains the \verb|rustc_ast| and \verb|rustc_parse| package which are responsible for this parsing.

However as Rust is a relatively new language many of these features are unstable. This means they are only available with the nightly compiler and the interface is likely to change. To avoid this, the syn package is used. "Syn is a parsing library for parsing a stream of Rust tokens into a syntax tree of Rust source code" \cite{syn}. This provides a stable interface for parsing and manipulating the AST.
The main issue with Syn, is it geared toward helping parse code in Rust procedural macros. This means it has a stronger focus on understanding an AST rather than manually manipulating it and creating items. For this reason a fork of the Syn project was created to allow access to an additional library, which made creating more AST objects possible. The fork can be found at \url{https://github.com/jelgar/syn}.

Another requirement, is once the AST has be transformed, it must be converted back into Rust code. To achieve this the \verb|quote| and \verb|rustfmt|. The \verb|quote|  package converts the AST back into tokens of source code, whilst the \verb|rustfmt| package formats the outputted code in a consistent way.

\section{Transformer Implementation}



% \begin{figure}[t]
% \centering
% foo
% \caption{This is an example figure.}
% \label{fig}
% \end{figure}

% \begin{table}[t]
% \centering
% \begin{tabular}{|cc|c|}
% \hline
% foo      & bar      & baz      \\
% \hline
% $0     $ & $0     $ & $0     $ \\
% $1     $ & $1     $ & $1     $ \\
% $\vdots$ & $\vdots$ & $\vdots$ \\
% $9     $ & $9     $ & $9     $ \\
% \hline
% \end{tabular}
% \caption{This is an example table.}
% \label{tab}
% \end{table}

% \begin{algorithm}[t]
% \For{$i=0$ {\bf upto} $n$}{
%   $t_i \leftarrow 0$\;
% }
% \caption{This is an example algorithm.}
% \label{alg}
% \end{algorithm}
% 
% \begin{lstlisting}[float={t},caption={This is an example listing.},label={lst},language=C]
% for( i = 0; i < n; i++ ) {
%   t[ i ] = 0;
% }
% \end{lstlisting}

\subsubsection{Example Sub-sub-section}

This is an example sub-sub-section;
the following content is auto-generated dummy text.

\paragraph{Example paragraph.}

This is an example paragraph; note the trailing full-stop in the title,
which is intended to ensure it does not run into the text.

% -----------------------------------------------------------------------------

\chapter{Critical Evaluation}
\label{chap:evaluation}

{\bf A topic-specific chapter} 
\vspace{1cm} 

\noindent
This chapter is intended to evaluate what you did.  The content is highly 
topic-specific, but for many projects will have flavours of the following:

\begin{enumerate}
\item functional  testing, including analysis and explanation of failure 
      cases,
\item behavioural testing, often including analysis of any results that 
      draw some form of conclusion wrt. the aims and objectives,
      and
\item evaluation of options and decisions within the project, and/or a
      comparison with alternatives.
\end{enumerate}

\noindent
This chapter often acts to differentiate project quality: even if the work
completed is of a high technical quality, critical yet objective evaluation 
and comparison of the outcomes is crucial.  In essence, the reader wants to
learn something, so the worst examples amount to simple statements of fact 
(e.g., ``graph X shows the result is Y''); the best examples are analytical 
and exploratory (e.g., ``graph X shows the result is Y, which means Z; this 
contradicts [1], which may be because I use a different assumption'').  As 
such, both positive {\em and} negative outcomes are valid {\em if} presented 
in a suitable manner.

% -----------------------------------------------------------------------------

\chapter{Conclusion}
\label{chap:conclusion}

\noindent
The concluding chapter of a dissertation is often underutilised because it 
is too often left too close to the deadline: it is important to allocate
enough attention to it.  Ideally, the chapter will consist of three parts:

\begin{enumerate}
\item (Re)summarise the main contributions and achievements, in essence
      summing up the content.
\item Clearly state the current project status (e.g., ``X is working, Y 
      is not'') and evaluate what has been achieved with respect to the 
      initial aims and objectives (e.g., ``I completed aim X outlined 
      previously, the evidence for this is within Chapter Y'').  There 
      is no problem including aims which were not completed, but it is 
      important to evaluate and/or justify why this is the case.
\item Outline any open problems or future plans.  Rather than treat this
      only as an exercise in what you {\em could} have done given more 
      time, try to focus on any unexplored options or interesting outcomes
      (e.g., ``my experiment for X gave counter-intuitive results, this 
      could be because Y and would form an interesting area for further 
      study'' or ``users found feature Z of my software difficult to use,
      which is obvious in hindsight but not during at design stage; to 
      resolve this, I could clearly apply the technique of Smith [7]'').
\end{enumerate}

% =============================================================================

% Finally, after the main matter, the back matter is specified.  This is
% typically populated with just the bibliography.  LaTeX deals with these
% in one of two ways, namely
%
% - inline, which roughly means the author specifies entries using the 
%   \bibitem macro and typesets them manually, or
% - using BiBTeX, which means entries are contained in a separate file
%   (which is essentially a databased) then inported; this is the 
%   approach used below, with the databased being dissertation.bib.
%
% Either way, the each entry has a key (or identifier) which can be used
% in the main matter to cite it, e.g., \cite{X}, \cite[Chapter 2}{Y}.
%
% We would recommend using BiBTeX, since it guarantees a consistent referencing style 
% and since many sites (such as dblp) provide references in BiBTeX format. 
% However, note that by default, BiBTeX will ignore capital letters in article titles 
% to ensure consistency of style. This can lead to e.g. "NP-completeness" becoming
% "np-completeness". To avoid this, make sure any capital letters you want to preserve
% are enclosed in braces in the .bib, e.g. "{NP}-completeness".

\backmatter

\bibliography{dissertation}

% -----------------------------------------------------------------------------

% The dissertation concludes with a set of (optional) appendicies; these are 
% the same as chapters in a sense, but once signaled as being appendicies via
% the associated macro, LaTeX manages them appropriatly.

\appendix

\chapter{An Example Appendix}
\label{appx:example}

Content which is not central to, but may enhance the dissertation can be 
included in one or more appendices; examples include, but are not limited
to

\begin{itemize}
\item lengthy mathematical proofs, numerical or graphical results which 
      are summarised in the main body,
\item sample or example calculations, 
      and
\item results of user studies or questionnaires.
\end{itemize}

\noindent
Note that in line with most research conferences, the marking panel is not
obliged to read such appendices. The point of including them is to serve as
an additional reference if and only if the marker needs it in order to check
something in the main text. For example, the marker might check a program listing 
in an appendix if they think the description in the main dissertation is ambiguous.

% =============================================================================

\end{document}
